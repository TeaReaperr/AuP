\documentclass[a4paper, 12pt, ngerman]{article}
\usepackage[utf8]{inputenc}
\usepackage[T1]{fontenc}
\usepackage{babel}
\usepackage{lmodern}

\begin {document}
	\begin{center} 
	\textbf{\textsl{Aufgabenblatt 01}}\\
	\end{center}
\textbf {Aufgabe 1}\\
\textbf1. \\
Ein Algorithmus ist ein Ablauf einer Funktion, der versucht ein Problem so effektiv wie
möglich zu lösen.\\
\textbf{2.} \\
Kaffeeautomat\\
Weg zur Uni\\
Essen in der Pfanne zubereiten\\
\textbf{3.}\\
\textbf{Kaffeeautomat}\\
- Wähle Getränk\\
- Werfe Geld ein\\
- \underline{if} zu viel Geld eingeworfen\\ 
 	\hspace*{1cm}Rückgeld zahlen\\
- \underline{else} kein Rückgeld zahlen \\
- \underline{if} kein eigener Becher\\
	\hspace*{1cm}Einwegbecher zur verfügung stellen\\
- \underline{else} eingestellten Becher erkennen\\
- \underline{while} Becher nicht voll\\
 	\hspace*{1cm}Becher Füllen\\

\noindent\textbf{Weg Zur Uni}\\
- Zum Zug laufen\\
- einsteigen\\
- \underline{while} nicht in Cottbus\\
 	\hspace*{1cm}Mit Zug fahren\\
- Zur Uni laufen\\

\noindent\textbf{Essen in der Pfanne zubereiten}\\
- Zutaten bereitstellen\\
- Pfanne auf den Herd stellen\\
- Herd anschalten\\
- Öl in die Pfanne geben\\
- Essen in die Pfanne geben\\
- \underline{while} essen nicht fertig gebraten\\
 	\hspace*{1cm}essen in der Pfanne lassen\\
\newpage

\noindent\textbf {Aufgabe 2}\\
\textbf{1.} \\
Reflexiv : \\
$\forall a \in A : (a,a)\in R$\\
Antisymmetrisch : \\
$\forall a,b \in A : (a,b) \in R, (b,a)\in R => a=b$\\
Transitiv : \\
$\forall a,b,c \in A : (a,b)\in R, (b,c)\in R => (a,c)\in R$\\
Asymmetrisch :\\
$\forall a,b \in A : (a,b)\in R => (b,a)\notin R$\\
Total :\\
$\forall a,b \in A : (a,b) \vee (b,a)\in R$\\
\textbf{2.} \\
Die erste Relation ist keine Äquivalenzrelation, weil für die Reflexivität zwei gleiche Zahlen
eingesetzt werden müssen und der Betrag aus Zahl 1 minus Zahl 2 immer 0.
Die zweite Relation ist keine Äquivalenzrelation, weil das Tupel (5,1) vorhanden sein kann,
aber nicht das Tupel (1,5) und somit die Symmetrie nicht gegeben ist.
Die letzte Relation ist eine Äquivalenzrelation, weil a=b ist und somit (a,a) für die reflexivität
immer vorhanden ist, kein (a,b) existieren kann und somit auch kein (b,a) für die Symmetrie
gebraucht wird und auch kein (a,b) und (b,c) existieren kann, womit (a,c) auch niemals
enthalten ist.

\end {document}