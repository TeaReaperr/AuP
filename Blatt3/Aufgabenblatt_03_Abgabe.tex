\documentclass[a4paper, 12pt, ngerman]{article}
\usepackage[utf8]{inputenc}
\usepackage[T1]{fontenc}
\usepackage{babel}
\usepackage{lmodern}

\begin {document}
	\begin {center}
	\textbf{Aufgabenblatt-03}
	\end {center}


\noindent \textbf{Aufgabe 1}\\
f1\\
Funktion\\
Wenn $a+b=c$ sein soll, aber a immer gleich bleibt und sich nur b  durch b1, 
b2 verändert muss c bei gleichbleibendem a ein anderes Ergebnis haben und 
somit gilt die Rechtseindeutigkeit.
f2\\
Funktion\\
Dadurch, dass $ax^2 + bx + c = dx^2 + ex + f$ sein muss, muss $a=d, b=e$ und 
$c=f$ gelten, weil sonst nicht garantiert der selbe Wert herauskommt.\\
f3\\
Keine Funktion\\
Nehmen wir an $f(b) = x^2 -1$ dann stimmt die Voraussetzung $f(b) = 0$, 
wenn $x = 1$ ist. Durch das Quadrat stimmt allerdings auch $x = -1$ und 1 $\neq$ -1\\
f4\\
Keine Funktion\\
Aufgrund der Rechtseindeutigkeit nehmen wir an, dass es ein Tupel $(a,b1)$ 
und ein Tupel $(a,b2)$ gibt, wobei b1 $\neq$ b2 ist. Zum Beispiel bin ich a, Yannick
ist b1 und Maik ist b2. Wir gehören alle zu der Menge S, da wir alle Studenten
sind, aber weil Yannick $\neq$ Maik ist gilt die Rechtseindeutigkeit nicht.\\
f5\\
Funktion\\
B ist die Menge von allen Studenten, die einen Studenten kennen und weil B in der 
Potenzmenge der Studenten vorhanden ist, sind auch alle Kombinationen davon vorhanden.\\

\noindent \textbf{Aufgabe 3}\\
1.\\
a = T - erstes Element von lösche\\
b = String - erstes Element aus T\\
c = Int - zwietes Element aus T\\
d = String - erstes Element aus T\\
e = Int - zweites Element aus T\\
f = TListe - zweites Element der nicht leeren Liste\\
g = T - erstes Element von ersetze\\
h = T - erstes Element der nicht leeren TListe\\
i = String - erstes Element von T\\
j = T - erstes Element der nicht leeren TListe\\
k = String - erstes Element von T\\
l = TListe - zweites Element aus nicht leerer Liste\\
m = Int - hat predecessor und kann == 0 sein\\
n = T - erstes Element nicht leere Liste\\
o = TListe - zweites Element aus nicht leere Liste\\

\newpage\noindent 2.\\
liste1\\
loesche (T Ernie 7) (NichtLeer (T Ernie 7) (NichtLeer (T Bert 9)\\
(NichtLeer (T Bibo 7) Leer)))\\
NichtLeer (T Bert 9) (NichtLeer (T Bibo 7) Leer)\\
\noindent\\
liste2\\
ersetze (T Bibo 9) (NichtLeer (T Ernie 8) (NichtLeer (T Bibo 7) Leer))\\
NichtLeer (T Ernie 8) (ersetze (T Bibo 9) (NichtLeer (T Bibo 7) Leer))\\
NichtLeer (T Ernie 8) (NichtLeer (T Bibo 9) (ersetze (T Bibo 7) Leer))\\
NichtLeer (T Ernie 8) (NichtLeer (T Bibo 9) Leer)\\
\noindent\\
liste3\\
ersetze (T Bibo 9) (NichtLeer (T Bibo 8) (NichtLeer (T Bibo 7) Leer))\\
NichtLeer (T Bibo 9) (ersetze (T Bibo 9) (NichtLeer (T Bibo 7) Leer))\\
NichtLeer (T Bibo 9) (NichtLeer (T Bibo 9) (ersetze (T Bibo 9)\\
(NichtLeer (T Bibo 7) Leer)))\\
NichtLeer (T Bibo 9) (NichtLeer (T Bibo 9) Leer)\\
\noindent\\
liste3\\
findeAnIdx 1 (NichtLeer (T Bibo 8) (NichtLeer (T Bibo 7) Leer))\\
findeAnIdx 0 (NichtLeer (T Bibo 7) Leer)\\
Doch (T Bibo 7)

\end {document}